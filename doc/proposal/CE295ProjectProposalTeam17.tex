\documentclass[12pt]{article}

\usepackage[top=1in, bottom=1in, left=1in, right=1in]{geometry} 
\usepackage{graphicx}
\usepackage{titling}
\usepackage{float}
\usepackage{bm}
%\usepackage[fleqn]{amsmath}
\usepackage{amssymb,amsmath}
\usepackage{listings}
\usepackage{color}
\usepackage{multirow}
\usepackage{enumitem}
\usepackage{fancyvrb}
\usepackage{hyperref}
\usepackage{setspace}
\usepackage{tabularx}
\usepackage{diagbox}
\geometry{letterpaper}
\linespread{1.1}% \geometry{landscape} % rotated page geometry

\definecolor{codegreen}{rgb}{0,0.6,0}
\definecolor{codegray}{rgb}{0.5,0.5,0.5}
\definecolor{codepurple}{rgb}{0.58,0,0.82}
\definecolor{backcolour}{rgb}{0.95,0.95,0.92}
\definecolor{outcolor}{rgb}{0.545, 0.0, 0.0}

\lstdefinestyle{mystyle}{
	backgroundcolor=\color{backcolour},   
	commentstyle=\color{codegreen},
	keywordstyle=\color{magenta},
	numberstyle=\tiny\color{codegray},
	stringstyle=\color{codepurple},
	basicstyle=\footnotesize,
	breakatwhitespace=false,         
	breaklines=true,                 
	captionpos=b,                    
	keepspaces=true,                 
	numbers=left,                    
	numbersep=5pt,                  
	showspaces=false,                
	showstringspaces=false,
	showtabs=false,                  
	tabsize=2
}

\lstset{style=mystyle}

\setlength{\droptitle}{-5em}
\title{CE 295 Project Proposal}
\date{9 Feb. 2018} 
\author{Team 17}

\begin{document}
	
	\maketitle
	\renewcommand\theequation{\arabic{equation}}
	\renewcommand{\figurename}{Fig.}
	\renewcommand\thesection{\Roman{section}}
	\renewcommand\thesubsection{(\alph{subsection})}
	%\onehalfspacing
	
\section{Title \& Team Member Names}
\textbf{Title:} States Estimation of Li-ion Batteries with Electro-Thermal-Aging Dynamics\\
\textbf{Team Member:} Xin Peng, Junzhe Shi, Franklin Zhao, and Ruitong Zhu
\section{Abstract}
Batteries are ubiquitous in all forms of electronics and transportation, and also are a key to the store of clean and secure energy. In different kinds of batteries,  Li-ion battery is the most prominent one due to their superior gravimetric and volumetric energy density.  For the safe operation of Li-ion battery, the state of charge (SOC) and state of health (SOH) estimation is of great significance. Hence, the goal of the project is to design a robust observer which can estimate the SOC and SOH of Li-ion batteries. In the project, the equivalent-circuit model will be used for the battery modeling with current and ambient temperature as inputs and voltage as the measured output. The equivalent-circuit model will include three parts which are an electrical model, a thermal model, and an aging model. To ensure the accuracy of states estimation, both of Luenberger observer and Kalman filter will be used and examined in the project. The battery system will be built and simulated by MATLAB and/or Python. The robustness of designed observer will be tested by real Li-ion battery testing data.
\section{Introduction}
\subsection{Motivation \& Background}
The identification of battery operation and aging in real life has been a long-desired yet challenging goal, which includes multiple complex processes in complicated operating conditions and environments. An accurate method to observe SOC and SOH of Li-ion battery is in need. Meanwhile, batteries invariably work at varying thermal and aging conditions. Thus, it’s necessary for us to build a battery observation system to monitor operation and aging of battery. The potential challenges also exist. We need to express a multi-control problem via mathematical equations and combine electrical model, thermal model and aging model. Besides, since all team members are major in Civil System, a lack of background knowledge in Electrical Engineering can be a big challenge. However, our previous course CE 291F, Control and Optimization of Distributed Parameters Systems can be helpful for the project. It gave us a background knowledge of partial differential equations, conservation laws, linear stability, Kaman filter and so on. We all have experiences of building, controlling and optimizing systems, including quench process,  heat diffusion and Lighthill Whitham Richards model.
\subsection{Focus of this study}
In this project, we will focus on the SOC and SOH of LI-ion batteries. Based on equivalent-circuit, the electrical, thermal and aging models will be developed for the observing system. Since battery monitoring and management can be the key to allowing innovation in future designs because of their limit properties, our system may play an important role in such an area, and significantly contribute to the energy saving and efficiency. 
\newpage
\section{Statement of Work}
The statement of work is shown in Table 1.
\begin{center}
Table 1: Statement of Work
\end{center}
\vspace{-23pt}
\begin{figure}[H]
	\centering
	\includegraphics[width=\linewidth]{tmp.jpg}      
\end{figure}
\newpage
\section{Summary}
This project will basically develop an obeserving system with high efficiency and robustness. Eletrical, thermal, and aging models will be built based on the equivalent-circuit model. The accuracy will be tested using Luenberger observer and Kalman filter.
\section*{Relevant Literature}
\begin{itemize}[noitemsep, topsep=0pt]
	\item[{[1]}] Perez, Hector, et al. ``Optimal Charging of Li-Ion Batteries with Coupled Electro-Thermal-Aging Dynamics." \textit{IEEE Transactions on Vehicular Technology}, vol. 66, no. 9, pp. 99, 2017.
	\item[{[2]}] G. L. Plett, ``Extended Kalman filtering for battery management systems of LiPB-based HEV battery packs: Part 1. Background," \textit{Journal of Power Sources}, vol. 134, no. 2, pp. 252 – 61, 2004.
	\item[{[3]}] G. L. Plett, ``Extended Kalman filtering for battery management systems of LiPB-based HEV battery packs: Part 2. Modeling and identification," \textit{Journal of Power Sources}, vol. 134, no. 2, pp. 262 – 76, 2004.
	\item[{[4]}] G. L. Plett, ``Extended Kalman filtering for battery management systems of LiPB-based HEV battery packs: Part 3. State and parameter estimation," \textit{Journal of Power Sources}, vol. 134, no. 2, pp. 262 – 76, 2004.
	\item[{[5]}] Lin, Xinfan, et al. ``A lumped-parameter electro-thermal model for cylindrical batteries." \textit{Journal of Power Sources}, vol. 257, no. 0, pp. 1 – 11, 2014.
\end{itemize}
\end{document}